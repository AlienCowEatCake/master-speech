\documentclass[a4paper,10pt]{article}

% Русский язык (LaTeX)
\usepackage[T1,T2A]{fontenc}
\usepackage[utf8]{inputenc}
\usepackage[english,russian]{babel}

% Междустрочный интервал
\usepackage[nodisplayskipstretch]{setspace}
\singlespacing
% Нет интервала между элементами списков
\usepackage{enumitem}
\setlist{nosep}
% Отступы в списках
\setitemize[0]{leftmargin=1cm, itemindent=0cm}
\setenumerate[0]{leftmargin=1cm, itemindent=0cm}

% Интервал между абзацами
\setlength{\parskip}{0pt}

% Поля страницы
\usepackage[left=3cm,top=2cm,right=1cm,bottom=2cm]{geometry}

% Отступ для абзаца
\setlength{\parindent}{0cm}

% Запрет висячих строк
\clubpenalty=10000
\widowpenalty=10000

% Структурированный pdf
\usepackage[bookmarksnumbered=true,bookmarksopen=true,breaklinks=true,pdfborder={0 0 0},linktoc=all]{hyperref}

% Минимальное количество букв, которые можно переносить
\righthyphenmin=2

% Борьба с overfull
\tolerance=300000

% Выключить нумерацию страниц
\pagestyle{empty}

% =============================================================================

\begin{document}

\subsection*{Слайд 1}
Здравствуйте, меня зовут Жигалов Петр Сергеевич. Тема моей магистерской диссертации -- <<Анализ систем источник-приемник в задачах морской геоэлектрики>>.

\subsection*{Слайд 2}
Целью работы является решение трёхмерной прямой задачи морской геоэлектрики векторным методом конечных элементов. В рамках поставленной цели были сформулированы следующие задачи:
\begin{itemize}
	\item Исследование влияния слоя воздуха при различной глубине источника электромагнитного возмущения.
	\item Исследование целесообразности применения PML-слоя для ограничения области моделирования на низких частотах.
	\item Исследование поведения электромагнитного поля при различном расположении источника поля и искомого объекта друг относительно друга.
\end{itemize}

\subsection*{Слайд 3}
В современном мире экономика многих стран зависит от цены на нефть, поэтому актуальными в последнее время становятся задачи геологоразведки в недрах земли, скрытых под толщей морской воды. Обычно задача морской геоэлектрики выглядит так, как показано на слайде: источник перемещается кораблём со специальным оборудованием, а приёмники располагаются на морском дне.

\subsection*{Слайд 4}
Задачи в морской геоэлектрике в частотной области описываются уравнением Гельмгольца. Это уравнение, а также краевые условия вы можете видеть на слайде.

\subsection*{Слайд 5}
На этом слайде представлена векторная вариационная постановка в пространстве $H(rot)$. В качестве конечных элементов используются тетраэдры.

\subsection*{Слайд 6}
Размеры области моделирования в задачах морской геоэлектрики составляют более 6000~м. Это приводит к необходимости применения методов для сокращения расчётной области, например идеально согласованного слоя или PML-слоя. Он представляет собой подобласть со специальными коэффициентами, построенными таким образом, чтобы обеспечить полное поглощение электрического поля внутри слоя и не допустить его отражения от внутренних границ.

\subsection*{Слайд 7}
PML-слой определяется заменой координат, которую можно видеть на слайде.

\subsection*{Слайд 8}
На этом слайде представлена вариационная постановка с учетом замены координат.

\subsection*{Слайд 9}
Перейдем к исследованиям. Для сокращения расчетной области нередко в область моделирования не включается воздух. Однако, так как в воздухе протекает гиперболический волновой процесс. Рассмотрим, насколько велико его влияние на электрическое в остальной области при различной величине заглубления источника поля. Расчетная область представлена на слайде. Будут рассматриваться две задачи, одна на полной области, другая на сокращенной, где на границе $\Omega_2$ и $\Omega_1$ задаются условия непротекания, а область $\Omega_1$ отсутствует.

\subsection*{Слайд 10}
На этом слайде представлена норма разности решений при различных заглублениях источника электромагнитного поля.

\subsection*{Слайд 11}
А на этом слайде представлена $y$-компонента электрического поля в грунте недалеко от поверхности. Видно, что процессы в слое воздуха оказывают значительное влияние на поле при расположении источника электромагнитного возмущения на глубине меньше трёхсот метров для рассмотренной конфигурации.

\subsection*{Слайд 12}
В следующем исследовании рассмотрим эффективности применения PML-слоя. В высокочастотных задачах с волноводами PML-слой широко используется, а для низкочастотных задач это рассматривается впервые. Расчетная область представлена на слайде. В ходе исследования проводится решение модельной задачи без PML-слоя и с PML-слоем с варьированием параметров.

\subsection*{Слайд 13}
Некоторые полученные результаты можно видеть на слайде. Видно, что применение PML-слоя позволило получить достаточно точные решения, однако не привело к уменьшению времени решения.

\subsection*{Слайд 14}
На этом слайде представлены картины электрического поля без PML-слоя и с PML-слоем. Одной из причин того, что время решения не сокращается, может быть то, что основное растяжение приходится на вещественные компоненты координат. Это приводит к «вытянутости» тетраэдров внутри PML-слоя и сильному ухудшению свойств матрицы СЛАУ и увеличению времени решения. Параллелепипедальные конечные элементы лишены подобного недостатка, однако, такие элементы не подходят для аппроксимации сколь-либо сложных областей. Для использования в одной сетке и тетраэдральных, и параллелепипедальных конечных элементов можно воспользоваться неконформными методами.

\subsection*{Слайд 15}
В этом исследовании рассмотрим поведение электрического поля при различном расположении источника этого поля относительно объекта в грунте. Расчетная область представлена на слайде. Будем рассматривать ситуацию с проводящим и непроводящим объектом.

\subsection*{Слайд 16}
На этом слайде представлены картины $z$-компоненты электрического поля при нулевом смещении.

\subsection*{Слайд 17}
На этом слайде представлены картины поля при смещении в 100~м. Четко стали видны границы проводящего объекта.

\subsection*{Слайд 18}
На этом слайде представлены картины поля при смещении в 200~м. На этот раз уже можно различить оба объекта.

\subsection*{Слайд 19}
Подведем краткие выводы:
\begin{itemize}
	\item Расчёты, в которых в область моделирования не включается воздух, допустимы только при расположении источника электромагнитного поля на большой глубине.
	\item Применение PML-слоя позволяет получить достаточно точные решения, однако его применение не приводит к резкому уменьшению времени решения.
	\item Проводящий объект хорошо <<виден>> на некотором расстоянии от морского дна, а непроводящий -- только вблизи дна или при небольшом заглублении приёмника в грунт. Наибольший отклик на источник электромагнитного возмущения для непроводящего объекта наблюдался в том случае, когда источник располагался со смещением от центра симметрии объекта.
\end{itemize}

\subsection*{Слайд 20}
Спасибо за внимание! Ваши вопросы, пожалуйста.


\end{document}
